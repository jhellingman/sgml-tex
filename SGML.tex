% SGML-TeX: make TeX understand (bare bones) SGML.
% copyright 1994, 1999 Jeroen Hellingman. This work may be distributed under the
% conditions of the GNU General Public License.
%
% TeX has been used as a typesetting engine for documents marked up with SGML. Mostly,
% this was accomplished with the use of a pre-processor, however, it is also possible
% for TeX to format documents with SGML mark-up directly. This file shows some simple
% TeX-hacks to demonstrate how TeX can parse SGML in a very limited way.
%
% These macros are not intended to produce CRC from SGML source, but only to have
% an acceptable output straight from SGML, such that works can be printed and
% proof read without the need of advanced (and expensive) tools.
%
% A more sophisticated solution will require a parser build in TeX, and is far more
% complicated than this, and a pre-processor may prove faster in that case.
%
%%%%%%%%%%%%%%%%%%%%%%%%%%%%%%%%%%%%%%%%%%%%%%%%%%%%%%%%%%%%%%%%%%%%%%%%%%%%

%%%% Start trickery %%%%%%%%%%%%%%%%%%%%%%%%%%%%%%%%%%%%%%%%%%%%%%%%%%%%%%%%
%
% Entities
%
% entity names are prefixed with :: to avoid clashing with TeX's
% macros. An entity &test; will call the macro \::test (which is inaccessible in
% Plain TeX.)

\def\SGMLentity#1#2{\expandafter\xdef\csname::#1\endcsname{\noexpand#2}}%
\def\SGMLamp#1;{\csname::#1\endcsname}

% Elements
%
% the macros for element names are prefixed with : to avoid clashing with TeX's
% macros. A tag <test> will call the macro \:test, and the tag </test> will call
% \:/test (which are both inaccessible from Plain TeX.)

\long\def\SGMLelement#1#2#3{\expandafter\xdef\csname:#1\endcsname{\noexpand#2}%
  \expandafter\xdef\csname:/#1\endcsname{\noexpand#3}}%

% Parsing of Tags
%
% The tag is the first word following the <, up to a space or >. As we only want
% to deal with a single case, we append a space and a special marker at the end
% of the tag.

\def\SGMLtag#1{\ifx#1!\skipTag#1\else\processTag#1\fi}
\long\def\skipTag#1>{\fi\relax}         % for some odd reason, this \fi is needed.
\def\processTag#1>{\spacedtag#1 @>}     % consume till closing >
\def\spacedtag#1 #2>{\parseAttrs#2@>\dotag{#1}\relax}   % consume till first space, flush rest
\def\dotag#1{\csname:#1\endcsname}      % call macro

% Parsing of Attributes
%
% In SGML, attributes can either have quotes or not. We handle both cases.
% Attributes are defined with ::: as prefix, and are accessible as long as
% they are not redefined, for example in a following element.
% attributes can be accessed by saying \theAttr{name} in macros

\def\parseAttrs#1#2>{\ifx#1@\else\parseAttr#1#2>\fi}
\def\parseAttr#1=#2#3>{\ifx#2"\parseQuotedAttr#1=#2#3>\else
    \parseUnquotedAttr#1=#2#3>\fi}
\def\parseQuotedAttr#1="#2" #3>{\setAttr#1=#2@>\parseAttrs#3>}
\def\parseUnquotedAttr#1=#2 #3>{\setAttr#1=#2@>\parseAttrs#3>}
\def\setAttr#1=#2@>{\expandafter\xdef\csname:::#1\endcsname{#2}}
\def\theAttr#1{\csname:::#1\endcsname}
\def\resetAttr#1{\xdef\csname:::#1\endcsname{{\bf[unknown]}\message{reset attribute accessed}}}
\def\setAttrTo#1#2{\xdef\csname:::#1\endcsname{#2}}

%%%%% changing catcodes

{\catcode`<=\active\gdef<{\SGMLtag}}
{\catcode`&=\active\gdef&{\SGMLamp}}

\def\SGML{%
  \catcode`\\=12    % make TeX specials normal characters
  \catcode`\#=12
  \catcode`\$=12
  \catcode`\%=12
  \catcode`\}=12
  \catcode`\{=12
  \catcode`\_=12
  \catcode`\^=12
  \catcode`\~=12
  \catcode`0=11\catcode`1=11\catcode`2=11\catcode`3=11\catcode`4=11%
  \catcode`5=11\catcode`6=11\catcode`7=11\catcode`8=11\catcode`9=11%
  \catcode`&=13     % make & and < active
  \catcode`<=13
}

\def\inputSGML #1 {\begingroup\SGML\input #1\endgroup}

% The following file defines a large number of common SGML entities.

\input SGMLentities

%%%% SGML elements %%%%%%%%%%%%%%%%%%%%%%%%%%%%%%%%%%%%%%%%%%%%%%%%%%%%%%%%%
%
% This is how to add TeX commands to a certain SGML element.
% When you define your own elements, take care of the following:
% - First define the macros you are going to use, and then call \SGMLelement.
% - Tags not defined will be ignored, the elements will have default lay-out.

\def\beginsgml{\relax}
\def\endsgml{\relax}
\SGMLelement{sgml}{\beginsgml}{\endsgml}

% this is how to skip (i.e. fold away) an unwanted element. Note that this will
% only work if the end tag is not omitted.

\def\skipElement{\bgroup\setbox0=\hbox\bgroup}
\def\endSkipElement{\egroup\relax\egroup}

\SGMLelement{skipElement}{\skipElement}{\endSkipElement}

% Accessing TeX
%
% we want to be able to use TeX constructs in our SGML document, and have them
% typeset as before. Before the closing </tex> you will have to say
% \SGML to revive the SGML conventions. (still looking for a trick to
% avoid this)

\def\beginTeX{% set the catcodes back to plain TeX conventions
  \catcode`&=4
  \catcode`\\=0
  \catcode`\#=6
  \catcode`\$=3
  \catcode`\%=14
  \catcode`\}=2
  \catcode`\{=1
  \catcode`\_=8
  \catcode`\^=7
  \catcode`\~=13
  \catcode`0=12\catcode`1=12\catcode`2=12\catcode`3=12\catcode`4=12%
  \catcode`5=12\catcode`6=12\catcode`7=12\catcode`8=12\catcode`9=12%
  \catcode`<=12}
\def\endTeX{\relax}
\SGMLelement{tex}{\beginTeX}{\endTeX}

%%%% eof %%%%%%%%%%%%%%%%%%%%%%%%%%%%%%%%%%%%%%%%%%%%%%%%%%%%%%%%%%%%%%%%%%%%
