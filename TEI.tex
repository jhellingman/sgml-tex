% TEI.tex -- Make TeX understand TEI formatted files (in a very limited way).
% copyright 1994, 1999 Jeroen Hellingman. This work may be distributed under the
% conditions of the GNU General Public License.
%
% NOTICE: this is a work in progress, and is still buggy in many ways. This
% approach will always be limited in what can be done, and depends on a number
% of conventions being used.
%
% known problems:
%   1. page breaks just after headers.
%   2. extra empty lines around paragraphs when they coincide with other tags.
%
% history:
%   13-AUG-1999 minor changes.
%   06-JUL-1999 changed <p> handling.
%   26-JUN-1999 made advance of footnote counter global (JH)

%% fonts I would like to use:

\font\rmVII=cmr7
\font\itVII=cmti7
\font\rmVIII=cmr8
\font\rmXII=cmr12
\font\bfXII=cmbx12
\font\bfXVII=cmbx12 at 17pt
\font\sc=cmcsc10

\input fontsize.tex
\tenpoint

\overfullrule=0pt
\parindent=0pt
\raggedbottom

%% load generic SGML macros

\input SGML.tex

%% Generally useful macros

\def\today{\the\day\ \ifcase\month\relax\or January\or
February\or March\or April\or May\or June\or July\or August\or
September\or October\or November\or December\fi\ \the\year}

\def\vandaag{\the\day\ \ifcase\month\relax\or januari\or
februari\or maart\or april\or mei\or juni\or juli\or augustus\or
september\or oktober\or november\or december\fi\ \the\year}

%% title page

\SGMLelement{docTitle}{\bgroup\bf}{\egroup\bigskip}

%% special parts of chapters

\SGMLelement{trailer}{\vfill}{\relax}

\SGMLelement{argument}{\medskip\bgroup\noexpand\ninepoint\advance\leftskip by1cm%
  \advance\rightskip by1cm}{\bigskip\egroup}

%% paragraphs and lines

%% The following is still a bit buggy. It would be best to give a \par at every
%% </p> tag, but, unfortunately, that tag is often omitted. We now use a small
%% trick to leave horizontal mode.
%\SGMLelement{p}{\vskip0pt\hskip0pt}{\relax}     % <p> will start a new paragraph

%% We rely on the source containing a double newline after each paragraph, to
%% obtain better formatting with TeX. Ignore the <p> tag for now.
%% Best would be to relay on the presence of </p>
\SGMLelement{p}{\relax}{\relax}

\SGMLelement{lg}{\medskip}{\medskip}
\SGMLelement{l}{\hfill\break}{\relax}

%% highlighting
%%
%% use \noexpand to allow later redefinition of \it to be visible after this
%% macro has been defined.

\SGMLelement{hi}{\bgroup\noexpand\it}{\/\egroup}

%% additional typographic tags, not from TEI, but used in intermediate files. Technically, we should
%% use the rend attribute, but that is difficult.

\SGMLelement{sc}{\bgroup\noexpand\sc}{\egroup}
\SGMLelement{b}{\bgroup\noexpand\bf}{\egroup}
\SGMLelement{i}{\bgroup\noexpand\it}{\egroup}

%% divisions and headers

\newcount\divZeroCount\divZeroCount=0
\newcount\divOneCount\divOneCount=0
\newcount\divTwoCount\divTwoCount=0
\newcount\divThreeCount\divThreeCount=0
\newcount\divFourCount\divFourCount=0
\newcount\divFiveCount\divFiveCount=0
\def\conditionalDivZeroCount{\ifnum\divZeroCount=0\relax\else\the\divZeroCount.\fi}
\newcount\divLevel\divLevel=0   % level of <div#> tag

\def\resetDivNumbers{\divZeroCount=0\divOneCount=0\divTwoCount=0\divThreeCount=0
  \divFourCount=0\divFiveCount=0}

\def\beginDivZero{\divLevel=0
  \global\advance\divZeroCount by1
  \global\divOneCount=0
  \global\divTwoCount=0
  \global\divThreeCount=0
  \global\divFourCount=0
  \global\divFiveCount=0}
\def\beginDivOne{\divLevel=1
  \global\advance\divOneCount by1
  \global\divTwoCount=0
  \global\divThreeCount=0
  \global\divFourCount=0
  \global\divFiveCount=0}
\def\beginDivTwo{\divLevel=2
  \global\advance\divTwoCount by1
  \global\divThreeCount=0
  \global\divFourCount=0
  \global\divFiveCount=0}
\def\beginDivThree{\divLevel=3
  \global\advance\divThreeCount by1
  \global\divFourCount=0
  \global\divFiveCount=0}
\def\beginDivFour{\divLevel=4
  \global\advance\divFourCount by1
  \global\divFiveCount=0}
\def\beginDivFive{\divLevel=5
  \global\advance\divFiveCount by1}

\def\endDiv{\advance\divLevel by-1}

\SGMLelement{div0}{\beginDivZero}{\endDiv}
\SGMLelement{div1}{\beginDivOne}{\endDiv}
\SGMLelement{div2}{\beginDivTwo}{\endDiv}
\SGMLelement{div3}{\beginDivThree}{\endDiv}
\SGMLelement{div4}{\beginDivFour}{\endDiv}
\SGMLelement{div5}{\beginDivFive}{\endDiv}

\SGMLelement{head}{\beginTeiHead}{\endTeiHead}

\def\beginTeiHead{\ifcase\divLevel \headZero\or\headOne\or\headTwo\or\headThree
    \or\headFour\or\headFive\or\headOther\fi}
\def\endTeiHead{\egroup\medskip\divLevel=6\nobreak} %% HACK: only first head after <div#> should be big
\def\headZero{\goodbreak\bigskip\bgroup\bfXVII\noindent
  \the\divZeroCount\ \ }
\def\headOne{\goodbreak\bigskip\bgroup\bfXVII\noindent
  \conditionalDivZeroCount\the\divOneCount\ \ }
\def\headTwo{\goodbreak\bigskip\bgroup\bfXII\noindent
  \conditionalDivZeroCount\the\divOneCount.\the\divTwoCount\ \ }
\def\headThree{\goodbreak\bigskip\bgroup\it\noindent
  \conditionalDivZeroCount\the\divOneCount.\the\divTwoCount.\the\divThreeCount\ \ }
\def\headFour{\goodbreak\medskip\bgroup\it\noindent}
\def\headFive{\goodbreak\medskip\bgroup\it\noindent}
\def\headOther{\medskip\goodbreak\bgroup\it\noindent}

%% notes in the margin

\newdimen\leftmarginwidth\leftmarginwidth=1.5cm

\def\leftnote#1{\leavevmode\vadjust{\handlebox{#1}}}
\def\handlebox#1{\vbox{\setbox1=\vtop{\kern-\baselineskip%
    \hsize=\leftmarginwidth\parindent=0pt%
    \advance\hsize by -0.25 cm\leftskip=-1pt%
    \raggedright\tolerance=1000#1}
  \dp1=0pt \ht1=0pt \llap{\hbox to \leftmarginwidth{\box1\hfil}}}}

%% original pagenumbers (placed in margin)

%\def\pageBreak{{ \bf[p.~\theAttr{n}] }\message{<\theAttr{n}>}}
\def\pageBreak{\leftnote{\rmVII p. \theAttr{n}}\message{<\theAttr{n}>}}

\SGMLelement{pb}{\pageBreak}{\relax}

%% notes in right margin

\newdimen\rightmarginwidth\rightmarginwidth=1.5cm

\def\rightnote#1{\leavevmode\vadjust{\handlerightbox{#1}}}
\def\handlerightbox#1{\vbox{\setbox1=\vtop{\kern-\baselineskip%
    \hsize=\rightmarginwidth\parindent=0pt%
    \advance\hsize by -0.25 cm\leftskip=-1pt%
    \raggedright\tolerance=1000#1}
  \dp1=0pt \ht1=0pt \rlap{\hbox to \hsize{\hfil\rlap{\kern0.25cm\box1}}}}}

%% corrections in the right margin

\def\correction{\rightnote{\rmVII[\theAttr{sic}]}}
\def\nocorrection{\rightnote{\itVII(sic)}}
\SGMLelement{corr}{\correction}{\noexpand\setAttrTo{sic}{\relax}}
\SGMLelement{sic}{\nocorrection}{\relax}

%% cross references (target placed in right margin)

\SGMLelement{ref}{\relax}{\rightnote{\noexpand\eightpoint$\rightarrow$ \sevenrm\theAttr{target}}}

%% footnotes

%% from the TeXbook:
\catcode`@=11
\def\footnote#1{\edef\@sf{\spacefactor\the\spacefactor}#1\@sf
 \insert\footins\bgroup\eightpoint
 \interlinepenalty100\let\par=\endgraf
 \leftskip=0pt\rightskip=0pt
 \splittopskip=10pt plus 1pt minus 1pt \floatingpenalty=20000
 \smallskip\item{#1}\bgroup\strut\aftergroup\@foot\let\next}
\skip\footins=12pt plus 2pt minus 4pt \dimen\footins=8in
\catcode`@=12

%% counting from the start of the document
\newcount\noteNumber\noteNumber=0
\def\note{\global\advance\noteNumber by1\footnote{$^{\the\noteNumber}$}\bgroup}

%% using the original note numbers
%\def\note{\footnote{$^{\theAttr{n}}$}\bgroup\rm}

\SGMLelement{note}{\note}{\egroup}

%% main parts of text

\SGMLelement{text}{\global\pageno=-1\resetDivNumbers\noteNumber=0\vfill\eject}{\relax}   % start text on new page
\SGMLelement{body}{\global\pageno=1\resetDivNumbers\noteNumber=0\vfill\eject}{\relax}
\SGMLelement{back}{\resetDivNumbers\noteNumber=0\vfill\eject}{\relax}

%% tables
%%
%% TODO: This doesn't handle <head> elements in <table> correctly.

\def\beginTable{\medskip\halign\bgroup####\hskip1em\hfill&&####\hskip1em\hfill\cr}
\def\endTable{\cr\egroup\medskip}
\def\beginRow{\cr}
\def\beginCell{&}

\SGMLelement{table}{\bgroup\beginTable}{\endTable\egroup}
\SGMLelement{row}{\beginRow}{\relax}
\SGMLelement{cell}{\beginCell}{\relax}

\def\tableHead{\noalign\bgroup\bf}

\def\beginList{\medskip}
\def\endList{\medskip}

\SGMLelement{list}{\beginList}{\endList}
\SGMLelement{item}{\item{$\bullet$}}{\relax}

%% The TEI header

\SGMLelement{teiHeader}{\hrule {\bf TEI header}\smallskip}{\smallskip\hrule}
\SGMLelement{publicationStmt}{\smallskip{\it Published by: }}{\smallskip}
\SGMLelement{availability}{\smallskip{\it Availability: }}{\smallskip}
\SGMLelement{sourceDesc}{\smallskip{\it Source: }}{\smallskip}
\SGMLelement{encodingDesc}{\smallskip{\it Encoding: }}{\smallskip}
\SGMLelement{revisionDesc}{\smallskip{\it Revision History: }}{\smallskip}
\SGMLelement{change}{\item}{\relax}

%% end
